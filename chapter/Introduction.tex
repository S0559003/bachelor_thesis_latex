\chapter{Einleitung}

\section{Motivation}

%Gartner HypeLifecycle
%https://upload-magazin.de/blog/34691-e-commerce-augmented-reality/
%Investitionenen in AR Fähige Endgeräte Microsoft Hololens, Apple Glass, NReal
[...]

\section{Zielsetzung}

Das Ziel dieser Arbeit ist es, ein System zu konzipieren und einen Prototypen zu entwickeln, welches dem Anwender ermöglicht, Feedback an Produkten
abzugeben. 

Das zu konzipierende System soll zum Einen Anwendungsszenarien unterstützen, in welchen Nutzer Feedback zu Produkten abgeben können, 
zum Anderen soll Nutzern ermöglicht werden, diese Feedbacks auf den Produkten zu explorieren. 
Diese Kommunikation soll eine Kollaboration von mehreren Nutzern, mit dem Produkt im Mittelpunkt, ermöglichen und zum Ziel haben, die Qualität der 
Produkte und die damit verbundene Kundenzufriedenheit zu verbessern.

Die prototypische Implementierung soll sich auf die Erstellung, Bearbeitung und Löschung von Feedback eingrenzen. Die Feedback sollen auf dem physischen 
Produkt in Form von Annotationen dargestellt und in einem im Web übertragbaren Format gespeichert werden. Dies soll den Zugriff auf diese Daten 
(z. B. für die Überführung in ein Anforderungsmanagement System) über eine Schnittstelle realisierbar machen.  

Nach der Implementierung des Prototypen soll die Usability für die Erstellung, Änderung und Löschung von Feedbacks auf einem Produkt durch 
eine Usability Studie evaluiert werden. 

\section{Vorgehensweise und Aufbau der Arbeit}

Zunächst werden im Kapitel \ref{CapterFundamentals} Grundlagen näher gebracht, die für das Verständnis dieser Arbeit benötigt werden. Anschließend wird 
im Kapitel 3 ein Überblick über den Stand der Technik im Bereich der Augmented Reality Technologien gegeben sowie Paradigmen für die Selektion in 3D Benutzeroberflächen, 
für die Anwendung auf mobilen Endgeräten wie Smartphones oder Tablets analysiert.  
 
Im Kapitel 4 wird die Konzeption des Systems beschrieben. Zunächst wird der Nutzungskontext des zu konzipierenden Systems analysiert und 
anschließend der Verlauf eines Kreativ Workshops beschrieben, welcher als Grundlage für die Anforderungsanalyse diente. Abschließend werden in diesem Kapitel 
die Anforderungen an den zu entwickelnden Prototypen beschrieben und ein Papierprototyp vorgestellt. 

Basierend auf den in den vorangegangenen zwei Kapitel erarbeiteten Konzeptionen und den dort definierten Anforderungen, wird im Kapitel 5 der Verlauf für die 
Umsetzung des digitalen Prototypen beschrieben.      

Im Kapitel 6 wird zunächst die Planung der durchzuführenden Nutzerstudie beschrieben. Dies beinhaltet die Festlegung des Studiendesigns sowie die Planung der Prozedur und die des Ablaufplans. 
Anschließend wird die Durchführung der Studie beschreiben sowie die Ergebnisse der erhobenen Daten vorgestellt. Zuletzt werden Folgerungen zu den aus der Studie erhobenen Daten diskutiert.

Abschließend wird im Kapitel 7 ein Fazit gezogen und die Arbeit zusammengefasst. Die Arbeit wird retrospektiv betrachtet . 
Zuletzt werden mögliche Erweiterungsmöglichkeiten der Anwendung beschrieben. 