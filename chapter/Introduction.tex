\chapter{Einleitung}

\section{Einleitung}
Diese Bachelorarbeit beschreibt den Entwurf und Implementierung einer Augmented Reality Anwendung. Die Anwendung soll  möglichst präzise und Aussagekräftige Rückmeldungen zu Gestaltung von Produkten ermöglichen.  Ziel der Anwendung  ist es, die Kommunikationsmöglichkeiten von Kundenrückmeldungen durch den Einsatz von Augmented Reality zu erforschen.

\section{Abstract}

\section{Motivation}


\section{Zielsetzung}

Das Ziel dieser Arbeit ist es ein System zu konzipieren und ein Prototypen zu entwickeln, welches dem Anwender ermöglicht Design- Rückmeldungen an Produkten %TODO Füge hier Fußnote ein das Design- Rückmeldungen ein bischen beschreibt
zu kommunizieren. 

Das zu konzipierende Gesamtsystems, soll zum einen Anwendungsszenarien unterstützen, in welcher Nutzer, Rückmeldungen zur Gestaltung von Produkten kommunizieren können, 
zum anderen soll Nutzern ermöglicht werden diese Rückmeldungen zu explorieren. 
Diese Kommunikation soll eine Kollaboration von mehreren Nutzern, mit dem Produkt im Mittelpunkt ermöglichen, und zum Ziel haben, die Qualität der Produkte und die damit 
verbundene Kundenzufriedenheit zu verbessern.

Diese Arbeit beschränkt sich auf das Erfassen von Kundenrückmeldungen. Die erfassten Informationen sollen die Grundlage für 
die Exploration und Analyse ermöglichen.

Nach einer prototypischen Implementierung soll am Ende, die Usability für die Erstellung, Änderung und Löschung von Design- Rückmeldungen an einem physischen Produkt, 
durch eine Usability Studie evaluiert werden. 

\section{Vorgehensweise und Aufbau der Arbeit}

Zunächst werden im Kapitel \ref{CapterFundamentals} Grundalgen näher gebracht und  
 
%Analyse: Stand der Technik, Ähnliche Arbeiten oder bestehende Systeme
% Konzeption: Workshop Nutzer und Aufgaben identifizieren, Personas und Szenarien entwicklen und daraus Anforderungen in form von User Stories ausarbeiten. 
% Usability Ziele festlegen
% Evaluation

