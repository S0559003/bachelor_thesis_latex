\chapter{Einleitung}

Diese Bachelorarbeit beschreibt den Entwurf und Implementierung einer Augmented Reality Anwendung. Die Anwendung soll  möglichst präzise und Aussagekräftige Rückmeldungen zu Gestaltung von Produkten ermöglichen.  Ziel der Anwendung  ist es, die Kommunikationsmöglichkeiten von Kundenrückmeldungen durch den Einsatz von Augmented Reality zu erforschen.

\section{Motivation}


\section{Zielsetzung}

Das Ziel dieser Arbeit ist es ein System zu konzipieren und ein Prototypen zu entwicklen, welches dem Anwender ermöglicht Änderungswünsche an Produkten 
zu komumizieren. Es soll mit aktuellen Möglichkeiten der Gemischten Realität (en. Mixed Reality (MR)) verwendet werden und 

% Wie muss Usability in MR gestaltet werden um möglichst effizient und effektiv Designvorschälge mitteilen zu können. 

% Vergleich mit bestehenden 

\section{Vorgehensweise und Aufbau der Arbeit}
TODO
%Analyse: Stand der Technik, Ähnliche Arbeiten oder bestehende Systeme
% Konzeption: Workshop Nutzer und Aufgaben identifizieren, Personas und Szenarien entwicklen und daraus Anforderungen in form von User Stories ausarbeiten. 
% Usability Ziele festlegen
% Evaluation

