\chapter{Einleitung}

\section{Motivation}


\section{Zielsetzung}

Das Ziel dieser Arbeit ist es, ein System zu konzipieren und einen Prototypen zu entwickeln, welches dem Anwender ermöglicht, Feedback an Produkten
abzugeben. 

Das zu konzipierende Gesamtsystem soll zum einen Anwendungsszenarien unterstützen, in welcher Nutzer Feedbacks zu Produkten abgeben können, 
zum anderen soll Nutzern ermöglicht werden, diese Feedbacks auf den Produkten zu explorieren. 
Diese Kommunikation soll eine Kollaboration von mehreren Nutzern, mit dem Produkt im Mittelpunkt, ermöglichen und zum Ziel haben, die Qualität der 
Produkte und die damit verbundene Kundenzufriedenheit zu verbessern.

Die prototypische Implementierung soll sich auf die Erstellung, Bearbeitung und Löschung von Feedbacks eingrenzen. Die Feedbacks sollen auf dem physischen 
Produkt in Form von Annotationen dargestellt, und in einer für im Web übertragbaren Format gespeichert werden. Dies soll den Zugriff auf diese Daten 
(z. Bsp. für die Überführung in ein Anforderungsmanagement System) über eine Schnittstelle realisierbar machen.  

Nach der Implementierung des Prototypen, soll die Usability für die Erstellung, Änderung und Löschung von Feedbacks an einem durch 
eine Usability Studie evaluiert werden. 

\section{Vorgehensweise und Aufbau der Arbeit}

Zunächst werden im Kapitel \ref{CapterFundamentals} Grundalgen näher gebracht, die für das Verständnis dieser Arbeit benötigt werden. Anschließend wird 
im Kapitel 3 ein Überblick über den Stand der Technik im Bereich der Augmented Reality Technologien gegeben sowie Paradigmen für die Selektion in 3D Benutzeroberflächen, 
insbesondere für Augmented Reality auf mobilen Endgeräten analysiert.  
 
Im Kapitel 4 wird die Konzeption des Gesamtsystems beschrieben. Zunächst wird der Nutzungskontext des zu konzipierenden Gesamtsystems analysiert und 
anschließend der Verlauf eines Kreativ Workshops beschrieben, welcher als Grundlage für die Anforderungsanalyse diente. Abschließend werden in diesem Kapitel 
die Anforderungen an den zu entwickelnden Prototypen beschrieben und ein Papierprototyp vorgestellt. 

Basierend auf den, in den vorangegangen zwei Kapitel erarbeiteten Konzeption und den dort definierten Anforderungen, wird im Kapitel 5 der Verlauf für die 
Umsetzung des digitalen Prototypen beschrieben.      

Im Kapitel 6 wird zunächst die Vorbereitung, auf die zu durchzuführende Usability Studie beschrieben. Dies beinhaltet die Festlegung der [...]

Abschließend wird im Kapitel 7 ein Fazit gezogen und die Arbeit zusammengefasst. Es wird in einer kritischen Rückblick erläutert was [...]. 
Zuletzt werden mögliche Erweiterungsmöglichkeiten der Anwendung beschrieben. 