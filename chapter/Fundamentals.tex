\chapter{Grundlagen}

Dieses Kapitel gibt eine  

\section{Virtual Reality} 

% Definition und Begriffseingrenzung von VR
Als Virtual Reality (z. dt. virtuelle Realität) werden 

% Anwendungsfelder von VR

% Schwachstellen und Probleme von VR

\section{Augmented Reality}

%Definition und Begriffseingrenzung  von AR
Abzugrenzen von VR  ist Augmented Reality (AR). [SchmalstiegHöllerer16] Im Gegensatz zu virtuellen Realität wo Benutzer vollständig in virtuelle Umgebungen eintauchen,
ist das Ziel bei AR, Informationen direkt in die physiche Ubgebung des Benutzers einzufügen. So soll der Eindruck erweckt werden, dass diese Informationen
Teil der wirklichen Welt sind. % Als Fußnote anmerken: Informationen müssen hierbei nicht nur auf visuelle Informationen beschränkt sein, sondern es kann sich dabei auch um auditive, haptische, gustative oder auch olfaktorisch Informationen sein.
[Azuma 1997] Während in VR, Benutzer von der Äußeren Umgebung nichts mitbekommen, wird mit AR Systeme die wirkliche Umebung des Benutzers, mit vitruellen Objekten überlagert. 
Azuma beschreibt folgende Charakteristiken welche AR Systeme aufweisen sollten: 

\begin{enumerate}
	\item Kombinieren Digitale und Reale Welt.
	\item Ermöglichen Interaktionen in Echtzeit.
	\item Informationen müssen sich 3D Raum befinden.
\end{enumerate}

Diese Characteristiken helfen dabei den Begriff Augmented Reality besser einzugrenzen zu können. [Azuma97 ] Filme wie z. Bsp.  ``Jurassic Park`` in welchen 
virtuelle Objekte in die reale 3D Szene eingefügt werden, erwecken den Eindruk dass diese Objekte Teil der realen Szene sind, jedoch kann mit diesen Objekten nicht in Echtzeit interagiert werden.

Ein anderes Beispiel ist im Vorschaufenster von Digitalkameras zu sehen. Oft blenden Digitalkameras im Vorschaubild, Informationen zum Ladezustand der Batterie, Aktivierung des Bilitzes und weitere 
Informationen bezüglich den aktuellen Einstellungen des Kameras ein. Diese Informationen überlagern zwar die realen Objekte in der Szene, befinden sich jedoch nicht im drei dimensionalen Raum. 
Der elektronishe Sucher hingegen welches Objekte (z. Bsp. Gesichter) erkennt und einrahmt befindet sich im 3D raum, und es sind Interaktionen in Echtzeit möglich. Indem sich das vom virtuellen  Object 
eingerahmte, real existierende Objekt oder die Kamera selbst sich bewegt, verändert sich auch die Position des virtuellen Objektes. 

[Füge hier Bild von Kamera mit Sucher und Einstellungen etc. ein]

% Motivation
[Azuma97] Durch das kombinieren von vitueller und physischer Welt erweitert Augumented Reality die Wahrnehmung des Menschen. Die Motivation von AR ist, den Menschen durch das Einfügen
von virtuellen Objekten in die physische Welt, Hinweise zu geben und Details zu zeigen die er mit seinen Sinnen sonst nicht unmittelbar wahnehmen könnte. Die Informationen die in die physische 
Welt hinzugefügt werden sollen den Menschen bei der Verrichtung von Aufgaben in der physischen Welt unterstützen.

% Anwendungsfelder von AR
Azuma fasst in [Azuma97]  Forschungen zu AR in sechs Anwendungsgebiete zusammen. Zur Visualisierung von Medizindaten, in der Wartung 
und Instandsetzung, Annotationen, für die Wegfindung für Roboter und für die Navigation von Militärflugzeugen. 
% Besonders passend für diese Arbeit sind Annotationen und Warung und Reperatur. Da diese sich besonders mit Interaktionen an Realen Objekten, meist Produkte befassen.

% Annotatinen
Beispielsweise können Annotatinen verwendet werden um Informationen über den Inhalt von Regalen einzublenden während ein Nutzer durch ein Bibliothek läuft. % Füge hier vielleicht noch ein Beispiel dazu ein % Füge hier Verweis auf ein Artike als Fußnotel hinzu
Auch können Annotationen in AR verwendet werden um einzelne Bauelemente an komplexen Bauteiteilen zu identifizieren und Informationen über diese zu geben. 

%Wartung und Instandsetzung
In der Wartung und Instandsetzung können Augemented Reality Anwendungen dabei helfen Instruktionen an komplexen Maschinen und Anlagen zu visualisieren welche sonst in 
Form von Text und Bildern vorliegen. So können virtuelle Replikate über die physischen Modelle gelegt, und so Schritt für Schritt Anleitungen direkt am physichen Produkt erstellt werden. 
Durch Animationen können diese Anleitungen mit zusätzlichen präziser gestaltet werden und zum Beispiel auch Informatinen  über die Richtung geben. 

Diese System können zum Beispiel    
% Remote Expert systeme 
%%TODO mach Verweis für Wartung (z. Bsp:https://www.ptc.com/-/media/Files/PDFs/Case-Studies/Howden-vuforia-studio-case-study-Feb-2019.pdf?la=en&hash=6342841E1B6470C1F313295427398606)
% IOT Sensordaten

%Grundlegende Techniken

% Schwachstellen und Probleme von AR
-  Probleme

\section{Mixed Reality}

% Definition von MR

% Anwendungsfelder MR

\section{Objekterkennung und- Verfolgung}

% Definition von Objekterkennung und Verfolgung

% Geschichte und Entwicklung

\subsection{Markerbasiertes Tracking}

% Begriffserklärung von Marker Trakting

% Anwendungsfelder von Marker Traking

\subsection{Markerloses Tracking}

% Begriffserklärung von Markerlosen Traking 

% Anwendungsfelder von Markerlosen Traking

\section{Usability}

Einen besonderen Fokus soll diese Arbeit auf die Usability legen. Daher wird in folgendem Abschnitt die Begriffsdefinition von Usablity näher beleuchtet, es werden einige 
gängige Methoden für die nutzerzentrierte Gestaltung und Entwicklung von Systemen vorgestellt und abschließend Methden für Usability Tests und Evaluirung erleutert. 

\subsection{Was ist Usability?}

% Begiffserklärung von Usability
In der Normreihe ISO 9241 welches als ein internationaler Standard,  Richtlinien für die Gestaltung von Mensch-Computer-Interaktionen beschreibt, wird im ISO Norm 9241-11,  Usability wie folgt definiert:

"das Ausmaß, in dem ein Produkt durch bestimmte Benutzer in einem bestimmten Nutzungskontext genutzt werden kann um bestimmte Ziele effektiv, effizient und zufriedenstellend zu erreichen."

% Usablity nicht nur Gestaltung der Nutzeroberfläche
[MichaelRichterMarkusFlückiger16; MaryBethRosson02] Usablity wird  oft als ein Qualitätskritärium für die Gestaltung der Benutzerschnittstelle verstanden. Dies ist jeodoch nicht ganz richtig.

%Die Benutzbarkeit eines Systems muss im Kontext seiner Verwendung beurteilt werden.[Michael Richter, Markus Flückiger]
Dass die Usability eines Systems nach dessen Nuzungskontext zu beurteilen ist verdeutlichen Michael Richter und Markus Flückiger an einem konkreten Beispiel für das Erfassung 
von Kurznachrichten (SMS) mit dem Aufkommen von Mobiltelefonen.  Bevor Smartphones mit Touchdisplays verbreitet waren, hatten Mobiltelefone oft rein 
numerische Tastaturen sodass, das Erfassen von Textnachrichten über die Nutzung der numerischen Tasten erfolgen musste. Indem zum Beispiel in kurzen Zeitabständen zwei mal
auf die Taste "2" gedrückt wurde, wurde zu Beispiel der Buchstabe ``B`` eingegeben. Diese Eingabemethode wurde oftmals von vielen Nutzern als umständlich empfunden. 
Jedoch konnte auf diese Weise effizient und zufridenstellend die die Aufgabe, eine Kurznachricht zu erfassen erfüllt werden. Zudem war diese Methode einfach zu erlernen und einprägsam.
 Somit wies diese Methode eine hohe Usablity auf. 

% Usablity nicht nur User friendly
[Nielsen94; Rex HartsonPardhaSPyla12] Oft wird Usablity auf die Eigenschaft eines Systems reduziert besonders benutzerfreundlich (en. User- friendly) zu sein. Der Begriff Usabliy umfasst jedoch mehr Aspekte. 

[Nielsen94] Mit dem Begriff "User- friendly" als Synonym für Usablity würde impliziert werden dass die Bedürfnisse von Benutzern mit einer einzigen Eigenschaft 
eines Systems beschrieben werden kann. In der Realität haben jedoch unterschiedlicheNutzer, unterschiedliche Bedürfnisse. Ein System welches zu einem Nutzer freundlich erscheint, könnte unter 
Umständen von einem anderen Nutzer als eher lästig empfunden werden.

% Eine Teilmenge der Systemazeptanz 
Nielsen Unterteilt Akzeptanzkriterien für ein Systems in soziale und praktische Kriterien.

Soziale bzw. ethische Akzeptanzkritärien sind solche, welche  die Nutzer von der Nutzung eines Sytems abhalten, selbst wenn praktische Akzeptanzkriterien eventuell sehr gut erfült werden. 
Spiekermann [vgl. Spiekermann 2016: 285] führt ein gutes Beispiel für ein solches Kriterium. Sie beschreibt am Beispiel eines Körperscanners in Flughäfen, dass trotzt  Berücksichtigung vieler 
praktischer Aspekte wie Ergonomie und trotz der effizienten und effektiven Aufgabenerfüllung ein solches System wenig Akzeptanz von den Nutzern haben kann. Beisielsweise fühlten sich 
Passagiere unangenehm wenn der Bildshirm auf welchem die nackten Umrisse ihrer Körper zu sehen war so platziert war dass andere Passagiere es auch sehen konnten.  % Füge hier Verweis als Fußnote zu (https://www.wired.com/2013/01/tsa-abandons-nude-scanners/)

Als praktische Kriterien führt er Eigenschafen wie Kosten, Kompatibilität, Zufverlässigkeit sowie Nutzbarkeit auf. Die Eigenschaft Nutzbarkeit teilt er in die Eigenschaften Nützlichkei (en. Utility) 
und Gebrauchstauglichkeit (en. Usability) auf. Unter Utility ist zu verstehen ob die Funktionalitäten eines Systems prinzipiell dazu in der Lage sind, die Aufgabe zu erfüllen wozu sie konzipiert wurden.

Die Eigenschaft Usability gliedert er in folgende fünf Teileigenschaften auf: 

\begin{itemize}
	\item Einfach zu erlernen.
	\item Effizient in der Nutzung.
	\item Leicht zu merken. (Ein Nutzer welcher das System einmal verwendet hat, sollte in der Lage sein nach einer längeren Pause das System zu nutzen ohne es erneut erlernen zu müssen.)
	\item Wenig Fehler. (Das System sollte wenig Fehler während der Nutzung zulassen. Im Falle das Fehler auftreten, im Falle das Fehler auftreten sollte es möglich sein dass sich das System 
	     	von diesem Fehler erholt und die Nutzung fortgeführt wird.)
	\item Subjektive Zufriedenstellung(Das System sollte angenehm zu nutzen sein. So dass Nutzer auch subjektiv zufriedengesellt werden während sie das System nutzen.)
\end{itemize}

%7 Kritärien nach ISO 9241 Teil 110 - ASSEFIL) 
Diese Kritärien sind im  ISO Norm  9241-110 als Grunsätze zur Dialoggestalung wie folgt aufgeführt: % Verweis als Fußnote auf % (https://www.medien.ifi.lmu.de/lehre/ss16/id/ISO_9241-10.pdf Liste Beispiele Aufgabenangemessenheit ab Seite 5)

\begin{itemize}
	\item Aufgabenangemessenheit
	\item Selbstbeschreibungsfähigkeit
	\item Steuerbarkeit
	\item Erwarungskonformität
	\item Fehlertoleranz
	\item Individualisierbarkeit
	\item Lernförderlichkeit
\end{itemize}

\subsection{Usablity Methoden}

[MichaelRichterMarkusFlückiger16]  Im laufe der Zeit haben sich verschiedene Fachrichtungen (wie z. Bsp: Human Computer Interaction (HCI), Human Factors, Interaction Design, Usability Engineering, 
User centered Design (UCD), User Experience (UX) und Design Thinking)  enwickelt welche nutzerorientierte Methoden für die Entwicklung von Technologien und neuen Anwendungen verfolgen. 

% Usablity Engineering
(MaryBethRosson et, al. 2002) Als eines dieser Fachrichtungen wurde die Fachrichtung Usablity Engineering von Usability Fachleuten bei Equipment Corparation 
ins Leben gerufen.  Der Begriff Usability Engineering steht für die Konzeption und Techniken für die Planun, Verifizierung und Abdeckung von Usability Zielen an ein System.  
Das Ziel von Usability Enginieering ist, messsbare Usability Ziele in den frühen Phasen des Softwareentwicklugsprozesses zu definieren und einen Rahmen zu schaffen diese 
Ziele im laufe der Entwicklung stätig überprüfen zu können um sicherstellen zu können dass diese Ziele erreicht werden.

Nielsen beschreibt in [Nielsen94] folgende Phasen im Lebenszyklus von Projekten mit Software Engenieering Methoden.

Kenne deine Nutzer:  

Im ersten Schritt werden alle Nutzer identifiziert, die mit dem System in Berührung kommen werden. Als Nutzer sollten in diesem Schritt alle Personen 
verstanden werden welche mit dem System oder mit Artefakten des Systems in Berührung kommen werden. Dies können Personen beinhalten welche das 
System installieren, konfigurieren, warten, administrieren oder Endkunden oder Kunden die das System selbst nie sehen werden jedoch Ergenisse von dem System erhalten werden.  
 
In einigen Fällen ist es einfacher potenzielle Nutzer von einem System zu identifizieren und deren Charakteristiken zu studieren. Zum Beispiel für ein Produkte die in einer bestimmten 
Abteilung einesbestimmten Unternehmens eingesetzt wereden soll. Schwieriger kann es hingegen für Produkte werden welche von einer breiteren Menge von Nutzern genutzt werden soll.

Es sollten Eigenschaten von Nutzern studiert werden welche für die Nutzen des Systems relevant sein könnten wie zum Bsp. Erfahrung von solchen Systemen und Enderäten, Bildungsstand, Alter. etc.
Dieser Schritt ist wichtig um die lernfährigkeit von Nutzern besser einschätzen zu können und so Kritärien für die kompleität der Nutzeroberfläche zu bestimmen.

% task analysis
Neben den Eigenschaften, sollten auch die Zile der Nutzer studiert. Wie bewältigen die  Nutzer mit aktuell Aufgaben um diese Zile zu erreichen. Hierbei sollte beobachtet werden welche Informationen 
die Nutzer benötigen, welche Ausnahme oder Not Situationen auftreten und wie die Nutzer in diesen Situationen handeln. Es sollte beobachtet werden ob die Nutzer das aktuell verwendete System in irgendweiner 
weise umgehen (en. Workarounds anwenden). Zudem sollten die Begrifflichkeiten notiert werden welche der Nutzer verwendet im Bezug auf die zu lösende Aufgabe verwendet.  
Diese können später als eine Quelle für Metapher bei der Gestaltung der Nutzeroberflähe verwendet werden. 

Schließlich sollten Schwachstellen untersicht werden 
% Schwachstellen des aktuellen problembewältigung

% functional analysis

% evolution of the user

Deiser Schritt ist wichtig um bestimte Charakteristiken von Nutzern zu studieren die relevant für die Gestalung der Komplexität 


Analyse bestender Produkte: 

		Usablity Ziele setzen  Beschreibung

		Paralle Gestaltung   Beschreibung

		Koordinierte Gestaltung der gesamten Benutzeroberfläche  Beschreibung 

		 Erstellung von Prototypen  Beschreibung 

		Empitisches testen  Beschreibung 

		Interative Gestaltung  Beschreibung 
		 Erhalte Rückmeldung von Nutzern  Beschreibung 


%User centered design

% Personas

% Use Cases

% Szenarios

\subsection{Usablity Tests und Evaluirung}


\section{Open Innovation}

% Definition von Open Innovatin vgl. zu Closed Innovation

% Anwendungsfelder

% Vorteile 

% Probleme
