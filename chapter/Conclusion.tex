\chapter{Fazit}

Dieses Kapitel fasst die Arbeit zusammen. Es wird mit Rückblick auf die Zielsetzung sowohl erreichtes reflektiert als auch Herausforderungen die sich gestellt haben und mögliche Verbesserungen hervorgehoben. 
Abschließend wird in einem Ausblick Erweiterungsmöglichkeiten beschrieben. 

\section{Zusammenfassung}

Im Kapitel \ref{anlayse_capter} dieser Arbeit wurden Forschungen im Gebiet der Informationsdarstellung in virtuellen Umgebungen (Abschnitt \ref{brand_abschnitt}) sowie Methoden für das 
Zeigen und Auswählen in AR Anwendungen (Abschnitt \ref{pointer_section}) betrachtet. Im Abschnitt \ref{ipi_section} wurden Arbeiten betrachtet welche ein Konzept für das in dieser Arbeit entworfene System vorstellen
und unterschiedliche Umsetzungsmöglichkeiten beschreibt. 

Im Kapitel \ref{conception_chapter} wurden im Rahmen eines Workshops abknöpfend an das im Kapitel \ref{CapterFundamentals} Abschnitt \ref{UsaEng} beschriebene Methode Usability Engineering, Nutzerprofile erstellt, die Aufgaben 
der Nutzer analysiert und Anforderungen für den Prototypen definiert. Darauf aufbauen wurde im Abschnitt \ref{lowfi} ein Papierprototyp erstellt. Dieser wurde einerseits genutzt um die Abdeckung der definierten Anforderungen 
zu überprüfen, sowie als grobe Vorlage für die Entwicklung des digitalen Prototypen im Kapitel \ref{implementation}. Zusätzlich zur dem Papierprototypen wurde als Grundlage für die Implementierung im Abschnitt \ref{objentwurf} ein 
ein Datentyp für ein Feedback definiert sowie ein Klassendiagramm erstellt. 

Im Kapitel \ref{implementation} wurden zunächst verschiedene zwei verschiedene Traking Framework und unterschiedliche Traking Methoden ausprobiert. 
Nachdem das Tracking mit Verwendung eines 3D Model des physischen Produktes nicht funktioniert hat wurde beschlossen ein Image Target für das Tracking mit Nutzung
des Vuforia Framework zu verwenden. 
Es wurden zwei Methoden für das Zeigen und Auswählen implementiert, von Test Nutzern getestet für das entschieden welches 




\section{Kritischer Rückblick}
TODO (Reflexion und Bewertung der Zielsetzung gegenüber erreichtem Ergebnis)

\section{Ausblick}
TODO