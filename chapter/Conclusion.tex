\chapter{Fazit}

Dieses Kapitel fasst die Arbeit zusammen. Es wird mit Rückblick auf die Zielsetzung sowohl Erreichtes reflektiert als auch Herausforderungen die sich während der Arbeit gestellt haben und mögliche Verbesserungen hervorgehoben. Abschließend werden in einem Ausblick Erweiterungsmöglichkeiten beschrieben. 

\section{Zusammenfassung}

Im Kapitel \ref{anlayse_capter} dieser Arbeit wurden Forschungen im Gebiet der Informationsdarstellung in virtuellen Umgebungen (Abschnitt \ref{brand_abschnitt}) sowie Methoden für das 
Zeigen und Auswählen in AR-Anwendungen (Abschnitt \ref{pointer_section}) betrachtet. Im Abschnitt \ref{ipi_section} wurden Arbeiten betrachtet, welche ein Konzept für das in dieser Arbeit entworfene System vorstellen
und unterschiedliche Umsetzungsmöglichkeiten beschreiben. 

Im Kapitel \ref{conception_chapter} wurden im Rahmen eines Workshops anknüpfend an die im Kapitel \ref{CapterFundamentals} Abschnitt \ref{UsaEng} beschriebene Methode des Usability Engineering, Nutzerprofile erstellt, die Aufgaben 
der Nutzer analysiert und Anforderungen für den Prototypen definiert. Darauf aufbauend wurde im Abschnitt \ref{lowfi} ein Papierprototyp erstellt. Dieser wurde einerseits genutzt um die Abdeckung der definierten Anforderungen 
zu überprüfen, andererseits auch als grobe Vorlage für die Entwicklung des digitalen Prototypen im Kapitel \ref{implementation}. Zusätzlich zum Papierprototypen wurde als Grundlage für die Implementierung im Abschnitt \ref{objentwurf} ein Datentyp für ein Feedback definiert sowie ein Klassendiagramm erstellt. 

Im Kapitel \ref{implementation} wurden zunächst verschiedene zwei verschiedene Tracking Frameworks und unterschiedliche Tracking Methoden ausprobiert. 
Nachdem das Tracking mit Verwendung eines 3D-Modell des physischen Produktes nicht funktioniert hat, wurde beschlossen ein Image Target für das Tracking mit Nutzung
des Vuforia Frameworks zu verwenden. 
Es wurden zwei im Kapitel \ref{anlayse_capter} behandelte Methoden für das Zeigen und Auswählen implementiert, diese wurden von Testnutzern getestet wurde die hinsichtlich ihrer Usability besser geeignete Methode in den Prototypen aufgenommen. Abschließend wurde im Kapitel \ref{implementation} ein Eingabeformular
gestaltet, welches mit Berücksichtigung der im Kapitel \ref{conception_chapter} identifizierten Anforderungen die Erstellung eines Feedback ermöglicht.  

Wie in der Zielsetzung vorgenommen, wurde im Kapitel \ref{study} der im Kapitel zuvor implementierte Prototyp hinsichtlich seiner Usability für die Aktionen: Erstellung, Bearbeitung und Löschung eines
Feedback evaluiert. Dabei hat sich ergeben, dass der entwickelte Prototyp für die Bearbeitung dieser Aufgaben eine hohe Usability aufweist, jedoch in Eigenschaften, welche in der Produktwahrnehmen eher 
in den Bereich des Nutzungserlebens (engl. User Experience, kurz UX) zugeordnet werden deutliches Verbesserungspotenzial aufweist. 

Zudem wurden in diesem Kapitel für die Bearbeitung von Feedback, zwei unterschiedliche Darstellungsformen verglichen und festgestellt wurde, dass 
die Darstellung der Feedbacks als Annotationen auf den Produkten im Vergleich zu einer Listenansicht zu mehr Zufriedenheit bei der Bearbeitung der Aufgabe führt uns sich daher besser eignet. 

Abschließend werden in diesem Kapitel die Arbeit hinsichtlich aufgetretenen Herausforderungen, mögliche Verbesserungen erläutert sowie in einem Ausblick Erweiterungsmöglichkeiten vorgestellt.  

\section{Kritischer Rückblick}

Reflektierend betrachtend kann hinsichtlich der Zielsetzung dieser Arbeit gesagt werden, dass diese mit der durchgeführten Nutzerstudie erreicht wurden.
Jedoch gab es einige Schwächen in der Umsetzung, welche in zukünftigen Arbeiten verbessert werden können. 

Zu diesen zählen organisatorische Schwächen, die sich auf folgende Punkte ausgewirkt haben:  

\begin{itemize}
\item{Für die Erarbeitung einer besser geeigneteren Lösung für die Überlagerung des physischen Produktes wurde nicht ausreichend Zeit eingeplant, sodass auf eine Lösung zurückgegriffen werden musste bei der 
die Überlagerung nicht optimal funktionierte.} 

\item{Das physische Produkt konnte für die Nutzerstudie nicht bereitgestellt werden. Dies hatte zur Folge dass die Darstellung als Annotation ohne eine Überlagerung des physischen Produktes und Listenansicht ohne die Anwesenheit des physischen Produktes getestet werden musste.}
\item{Für die Zeiterfassung für der Aufgabenbearbeitung wurde eine Lösung gewählt, die zwar getestet wurde und als sichere Methode empfunden wurde, jedoch 
keine geschickte Lösung darstellte. Mit dieser Zeiterfassungsmethode musste bei jeder Iteration der Aufgabenbearbeitung die Anwendung neugestartet werden, damit die Zeit für die nächste Aufgabe zurückgesetzt wurde.} 
\end{itemize}


\section{Ausblick}

Der entwickelte Prototyp kann in unterschiedliche Richtungen erweitert werden. Zum Einen können weitere in Kapitel \ref{conception_chapter} Abbildung \ref{img:sysstem_sketch} dargestellten Anforderungen implementiert 
werden. Wie zum Beispiel die Behandlung von Überdeckung, wichtiger Produktteile oder Annotationen. Eine Lösung für das Filtern und die in Clustern zusammenfügen der Annotationen (siehe Beispiel auf Abbildung \ref{img:annotation_clutter}) auf dem Produkt. Es könnte mit Hilfe der auf den Endgeräten verfügbaren Sensoren, dem Feedback weitere Informationen wie zum Beispiel Luftfeuchtigkeit, Helligkeit oder Bilder von den auswählten Stellen hinzugefügt werden. Dies würde die Aussagekraft der Feedbacks mit der Stärke der Darstellungsform Embedded Visualization nämlich (Eigenschaften aus der unmittelbaren Umgebung wahrnehmen zu können) bis zu einem bestimmten Grad stärken. 

Zum Anderen kann der Prototyp wie auf dem von \citeauthor{Kirschner2012} auf Abbildung \ref{img:objekt_centered_ipi} vorgestellten Konzept der objektzentrierten Ansatz der IPI Umsetzung, erweitert werden.
Die Grundlage für den Datenaustausch wurde mit der Erarbeitung eines Datentyps für Feedback und dessen Serialisierung und Deserialisierung erreicht. Im weiteren Schritt können diese Daten zwischen mehreren
Clients ausgetauscht und synchronisiert werden, sodass, wie auf Abbildung \ref{img:objekt_centered_ipi} vorgestellt, ein Informationsausgleich stattfindet und eine asynchrone Kommunikation über das Feedback ermöglicht wird. 

Zudem kann eine Schnittstelle erarbeitet werden mit der Hersteller auf diese Daten in strukturierter und gewichteter Form zugreifen und Erkenntnisse für neue Produktgenerationen gewinnen können.  




