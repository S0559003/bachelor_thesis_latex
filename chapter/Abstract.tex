\section*{Zusammenfassung}

Diese Bachelorarbeit beschreibt den Entwurf und die Implementierung einer Augmented Reality Anwendung für die Android Plattform. Die Anwendung soll möglichst präzises und aussagekräftiges Feedback zu Produkten ermöglichen.
Ziel der Anwendung ist es, die Kommunikationsmöglichkeiten von Kundenrückmeldungen durch den Einsatz von Augmented Reality zu erforschen.

Durch die Anwendung wird die Abgabe von Feedback in Form von Annotationen zu bestimmten Stellen oder Teilen am Produkt ermöglicht. Dazu wird das real existierende Produkt mit einem 
virtuellen 3D-Modell dieses Produktes überlagert. Der Nutzer kann mit diesem virtuellen Modell interagieren und Feedback erstellen, bearbeiten oder löschen. Die so abgegeben Feedbacks werden 
in einem im Web übertragbaren Format gespeichert, sodass die Grundlage für eine Kommunikation über dieses Feedback der Zugriff auf diese Daten (z. B. für die Überführung in ein Anforderungsmanagement-System) 
ermöglicht wird. Es werden unterschiedliche Methoden für das Zeigen auf dem Produkt, Auswählen bestimmter Stellen sowie unterschiedliche Darstellungsformen für die Informationsvisualisierung in Augmented Reality 
behandelt. 

Bereits vorhandene Konzepte für ein solches System werden betrachtet, eine Nutzungskontextanalyse durchgeführt und anschließend in einer Anforderungsanalyse Anforderungen an das zu entwickelnde System identifiziert. 
Nach der Implementierung eines Prototypen wird in einer Nutzerstudie die Usability des Prototypen hinsichtlich der Aktionen: Erstellung, Bearbeitung und Löschung von Feedback evaluiert sowie zwei unterschiedliche
Darstellungsformen hinsichtlich ihrer Effizienz und Zufriedenstellung in der Aufgabenbearbeitung verglichen.

\section*{Abstract}

This bachelor thesis describes the design and implementation of an augmented reality application for the Android platform. The application should enable more accurate and meaningful feedback on products.
The aim of the application is to explore the communication possibilities of customer feedback through the use of augmented reality.

Through the application, the dispensing of feedback in the form of annotations to specific points or parts of the product is supported. For this, the real existing product is overlayed with a
3D virtual model of this product. The user can interact with this virtual model and create, edit or delete feedback. The feedback will be
stored in a web-style format, providing the basis for communication via this feedback (e.g. for transfer to a requirements management system).
There are various methods for pointing to the product, selecting specific locations as well as various forms of information visualization in Augmented Reality
treated.

Existing concepts for such a system are taken into account, then a usage context analysis is performed in a requirement analysis, after which the needs for the system to be developed are identified.
After implementing a prototype, a user study evaluates the usability of the prototype based on the actions: creating, editing and deleting feedback, as well as two different
representation forms their efficiency and satisfaction in the task processing.
