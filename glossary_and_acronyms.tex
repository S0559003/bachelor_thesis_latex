\acrodef{PTAM}[PTAM]{\emph{Example}}
\newglossaryentry{api}
{
	name={API},
	description={Kurzbeschreibung für Application Programmer Interface}
}
\newglossaryentry{ar}
{
	name={AR},
	description={Kurzbeschreibung für Augmented Reality (zu dt. Erweiterte Realität)}
}

\newglossaryentry{vr}
{
	name={VR},
	description={Kurzbeschreibung für Virtual Reality (zu dt. Virtuelle Realität)}
}
\newglossaryentry{slam}
{
	name={SLAM},
	description={sample 4}
}
\newglossaryentry{ptam}
{
	name={PTAM},
	description={sample 4}
}

\newacronym[longplural={Frames per Second}]{fpsLabel}{FPS}{Frame per Second}

\newglossaryentry{gameengine}{name={Game Engine},description={sample 3}}
\newglossaryentry{serialisierung}{name={Serialisierung},description={sample 4}}
\newglossaryentry{deserialisierung}{name={Deserialisierung},description={sample 4}}
\newglossaryentry{klassendiagramm}{name={Klassendiagramm},description={sample 4}}

\longnewglossaryentry{framework}
{
	name=Framework
}
{Rahmenstruktur, Programmgerüst welches in der Softwaretechnik verwendet wird.}

\longnewglossaryentry{sdk}
{
	name=SDK
}
{Rahmenstruktur, Programmgerüst welches in der Softwaretechnik verwendet wird.}

\longnewglossaryentry{visualisierungspipeline}
{
	name=Visualisierugspipeline
}
{Rahmenstruktur, Programmgerüst welches in der Softwaretechnik verwendet wird.}

\longnewglossaryentry{cad}
{
	name=CAD
}
{Rahmenstruktur, Programmgerüst welches in der Softwaretechnik verwendet wird.}